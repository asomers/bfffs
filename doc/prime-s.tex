\documentclass[onecolumn,draft]{IEEEtran}
\usepackage{mathtools}
\usepackage{anysize}
\usepackage{listings}
\usepackage{wrapfig}
\usepackage[table]{xcolor}

\DeclarePairedDelimiter\floor{\lfloor}{\rfloor}

\title{SMR-compatible Declustered Disk Arrays}
\author{Alan Somers}

\begin{document}
\maketitle

\section{Background}

In 1990 Muntz and Lui~\cite{declust1} proposed declustered RAID.  It differs
from traditional RAID in that the encoding stripe size is less than the number
of disks in a cluster.  It offers several advantages, chiefly better
performance while rebuilding after a failure.  In 1992 Holland and
Gibson~\cite{sixrules} identify six conditions that any declustering algorithm
should ideally satisfy.  However, they did not develop an algorithm that
satisfied all six.  In 1998 Alvarez \emph{et al}~\cite{relpr} proved that all
six rules are impossible to satisfy simultaneously, except in a few special
cases.  They also presented two layout algorithms: PRIME and RELPR.  PRIME
satisfied five of the conditions and nearly satisfied the sixth: maximal read
parallelism.  RELPR satisfied four.  In addition to narrowly missing maximal
read parallelism, RELPR failed to satisfy fully distributed reconstruction.
But for most layouts, it comes close.  RELPR also achieved another goal: it
could work with any size cluster, any size of parity stripe, and any degree of
redundancy.

Spinning hard drives' performance has always been sensitive to sequential
access.  Each seek operation can take several milliseconds.  So it has always
gone without saying that any good declustering layout should minimize seeks by
placing logically sequential LBAs on the disks.  That is, user LBAs that are
near to each other should be mapped to physical LBAs that are also near to each
other.  However, most disks can handle very small seeks with no noticeable
impact on performance.  In addition, modern hard disks can use command queueing
to reorder adjacent but out-of-order operations, which can completely eliminate
the performance penalty for out-of-order access.  Even before hard drives
supported command queueing, operating systems could reorder accesses in
software.  So there was never any strong in incentive for declustering
algorithms to guarantee fully sequential access.

\section{Shingled Magnetic Recording}

Shingled Magnetic Recording (SMR) upends that assumption.  Host-managed SMR hard
drives~\cite{zbc}~\cite{zba} require that writes must be both adjacent and
sequential.  Out-or-order writes will be rejected as errors.  Thus, any
declustering layout algorithm intended for SMR hard drives must guarantee
complete adjacent and monotonic LBA mapping.  We can add a seventh condition to
the six from ~\cite{relpr}

\begin{enumerate}
  \item[7.] Monotonic mapping: The physical LBAs of each disk must monotonically
  map to logical stripes.
\end{enumerate}

Since no stripe can have more than one stripe unit on a single disk (condition
1), the reverse is guaranteed: logical stripes must map monotonically to
physical LBAs.

PRIME and RELPR come close to meeting condition 7.  All data units do map
monotonically.  However, check units do not.  A large write to a PRIME or RELPR
array will therefore contain small out-of-order sequences.

\section{PRIME-S}

PRIME is nearly monotonic.  In fact, it can be made completely monotonic simply
by sorting the stripe units within each iteration of the layout, since none of
the original six conditions places specific requirements on stripe units'
offsets. For example, \cite{relpr} gives this layout:

\begin{center}
\begin{tabular}{ |c|c|c|c|c| }
  \hline
  D0.0 & D0.1 & D1.0 & D1.1 & D2.0 \\
  \hline
  D2.1 & D3.0 & D3.1 & D4.0 & D4.1 \\
  \hline
  \rowcolor{lightgray} C4.0 & C2.0 & C0.0 & C3.0 & C1.0 \\
  \hline
  \rowcolor{lightgray}C1.1 & C4.1 & C2.1 & C0.1 & C3.1 \\
  \hline
  D5.0 & D6.1 & D5.1 & D7.0 & D6.0 \\
  \hline
  D7.1 & D9.0 & D8.0 & D9.1 & D8.1 \\
  \hline
  \rowcolor{lightgray}C9.0 & C8.0 & C7.0 & C6.0 & C5.0 \\
  \hline
  \rowcolor{lightgray}C6.1 & C5.1 & C9.1 & C8.1 & C7.1 \\
  \hline
  D10.0 & D11.0 & D12.0 & D10.1 & D11.1 \\
  \hline
  D12.1 & D13.1 & D14.1 & D13.0 & D14.0 \\
  \hline
  \rowcolor{lightgray}C14.0 & C10.0 & C11.0 & C12.0 & C13.0 \\
  \hline
  \rowcolor{lightgray}C11.1 & C12.1 & C13.1 & C14.1 & C10.1 \\
  \hline
  D15.0 & D17.0 & D16.1 & D16.0 & D15.1 \\
  \hline
  D17.1 & D19.1 & D19.0 & D18.1 & D18.0 \\
  \hline
  \rowcolor{lightgray}C19.0 & C16.0 & C18.0 & C15.0 & C17.0 \\
  \hline
  \rowcolor{lightgray}C16.1 & C18.1 & C15.1 & C17.1 & C19.1 \\
  \hline
\end{tabular}
\end{center}

Sorting each disk's stripe units by stripe, within each iteration, yields the
following monotonic layout:

\begin{center}
\begin{tabular}{ |c|c|c|c|c| }
  \hline
  D0.0 & D0.1 & \cellcolor{lightgray}C0.0 & \cellcolor{lightgray}C0.1 & \cellcolor{lightgray}C1.0 \\
  \hline
  \cellcolor{lightgray}C1.1 & \cellcolor{lightgray}C2.0 & D1.1 & D1.1 & D2.0 \\
  \hline
  D2.1 & D3.0 & \cellcolor{lightgray}C2.1 & \cellcolor{lightgray}C3.0 & \cellcolor{lightgray}C3.1 \\
  \hline
  \cellcolor{lightgray}C4.0 & \cellcolor{lightgray}C4.1 & D3.1 & D4.0 & D4.1 \\
  \hline
  D5.0 & \cellcolor{lightgray}C5.1 & D5.1 & \cellcolor{lightgray}C6.0 & \cellcolor{lightgray}C5.0 \\
  \hline
  \cellcolor{lightgray}C6.1 & D6.1 & \cellcolor{lightgray}C7.0 & D7.0 & D6.0 \\
  \hline
  D7.1 & \cellcolor{lightgray}C8.0 & D8.0 & \cellcolor{lightgray}C8.1 & \cellcolor{lightgray}C7.1 \\
  \hline
  \cellcolor{lightgray}C9.0 & D9.0 & \cellcolor{lightgray}C9.1 & D9.1 & D8.1 \\
  \hline
  D10.0 & \cellcolor{lightgray}C10.0 & \cellcolor{lightgray}C11.0 & D10.1 & \cellcolor{lightgray}C10.1 \\
  \hline
  \cellcolor{lightgray}C11.1 & D11.0 & D12.0 & \cellcolor{lightgray}C12.0 & D11.1 \\
  \hline
  D12.1 & \cellcolor{lightgray}C12.1 & \cellcolor{lightgray}C13.1 & D13.0 & \cellcolor{lightgray}C13.0 \\
  \hline
  \cellcolor{lightgray}C14.0 & D13.1 & D14.1 & \cellcolor{lightgray}C14.1 & D14.0 \\
  \hline
  D15.0 & \cellcolor{lightgray}C16.0 & \cellcolor{lightgray}C15.1 & \cellcolor{lightgray}C15.0 & D15.1 \\
  \hline
  \cellcolor{lightgray}C16.1 & D17.0 & D16.1 & D16.0 & \cellcolor{lightgray}C17.0 \\
  \hline
  D17.1 & \cellcolor{lightgray}C18.1 & \cellcolor{lightgray}C18.0 & \cellcolor{lightgray}C17.1 & D18.0 \\
  \hline
  \cellcolor{lightgray}C19.0 & D19.1 & D19.0 & D18.1 & \cellcolor{lightgray}C19.1 \\
  \hline
\end{tabular}
\end{center}

The offset function is more complicated for PRIME-S than for PRIME.  To
calculate it, we observe that PRIME-S's data units are placed in the same
posititon as PRIME's, but incremented by the number of earlier stripes' check
units that map to the same disk within that iteration.  And PRIME-S's check
units are placed in the same position as PRIME's data units from suceeding
stripes, but also incremented by the number of earlier stripes' check units
that map to the same disk within that iteration.  Exactly $f$ check units are
placed on each iteration of each disk, and we need to check all $f$.  If $s$ is
the stripe number and $b$ is the unit's position within its stripe, then the
offset formula thus becomes:

\begin{align}
	s &= \floor*{\frac{a}{m}} \\
	b &= \begin{cases}
		a - s m & \text{for data units} \\
		m + i & \text{for check units}
	\end{cases} \\
	\text{offset} &= \floor*{\frac{s m + b}{n}} + f z +
	\displaystyle\sum_{i=0}^{f} \begin{cases}
		1 & \text{if } s > \big((\text{disk } y^{-1} - i) - 1\big) \bmod n \\
		0 & \text{otherwise}
	\end{cases}
\end{align}

The disk formula is the same as for PRIME.  But we can unify PRIME's \emph{disk} and \emph{check-disk} functions like this:
\begin{align}
	disk = (s m + b) y \bmod n
\end{align}

Inverting PRIME-S is harder than inverting PRIME.  I don't know of a way that
doesn't involve some type of sort function.  However, most file system
operations don't actually need to invert the transform of a single stripe unit
very often or at all.  Rather, what they need is to determine all the stripes
that map to a range of LBAs on a given disk.  That can be done fairly
efficiently.  The stripes that are mapped to a given disk are determined by:

\begin{align}
	\text{stripes} = \{ i y^{-1} m^{-1} \bmod n\text{, } i \in [n - k, n) \}
\end{align}

\begin{thebibliography}{99}

\bibitem{declust1} Muntz, Richard R., and John CS Lui. ``Performance analysis
of disk arrays under failure.'' \emph{Proceedings of the 16th VLDB Conference},
pp 162-73, 1990.

\bibitem{schwabe1} Schwabe, E. and Sutherland, I.  ``Improved
parity-declustered layouts for disk arrays.'' \emph{Proceedings of the
Symposium on Parallel Algorithms and Architectures}, pp 76-84, 1994.

\bibitem{sixrules} Holland, Mark and Gibsom, Garth A. ``Parity declustering for
continuous operation in redundant disk arrays.'' \emph{Vol. 27, No. 9, ACM},
1992.

\bibitem{relpr} Alvarez, Guillermo A., et al. ``Declustered disk array
architectures with optimal and near-optimal parallelism.'' \emph{ACM SIGARCH
Computer Architecture News.} Vol. 26. No. 3. IEEE Computer Society, 1998.

\bibitem{zbc} TODO
\bibitem{zba} TODO

\end{thebibliography}

\section{Appendix}

\subsection{Inverting PRIME}

Alvarez et al~\cite{relpr} give the forward transforms for PRIME and RELPR, but
not the reverse transforms.  That is, they don't give an algorithm for
determining which stripe unit is placed at a given location on disk.  For
practical usage, that transform is needed for operations like defragmentation,
garbage collection, and failure handling.  The algorithm is:

\begin{align}
	z &= \floor*{\frac{\text{offset}}{kn}} \\
	y &= \big(z \bmod \left(n - 1\right) \big) + 1 \\
	a &= (\text{disk } y^{-1}) \bmod n + n ( \text{offset} - f z )\text{, for data units} \\
	i &= \text{offset} - (z + 1) k + f \\
	a &= m \Big(n z + \big((\text{disk } y^{-1} - i) m^{-1} - 1\big) \bmod n\Big),\text{ for check units}
\end{align}


\end{document}
